\documentclass[11pt]{article}
\usepackage{amsmath, amssymb, amsthm}
\usepackage{graphicx}
\usepackage{hyperref}
\usepackage{geometry}
\geometry{margin=1in}

\newtheorem{theorem}{Theorem}
\newtheorem{lemma}[theorem]{Lemma}
\newtheorem{corollary}[theorem]{Corollary}
\newtheorem{observation}[theorem]{Observation}
\newtheorem{conjecture}[theorem]{Conjecture}
\newtheorem{definition}[theorem]{Definition}
\theoremstyle{remark}
\newtheorem{remark}[theorem]{Remark}

\title{Signed Geometry of MHV Gravity: \\ The Forest Sign Rule and Split Signature}
\author{[Author Names]}
\date{\today}

\begin{document}

\maketitle

\begin{abstract}
We analyze the sign structure of the forest expansion of MHV gravity amplitudes. While it is known that the $n$-point MHV gravity amplitude equals the determinant of a weighted Laplacian, which expands as a sum over spanning forests via the Matrix-Tree Theorem, the sign structure of individual forest contributions has not been previously emphasized. We show that each forest term has sign $\varepsilon(F) = \text{sign}(\prod_e w_e) \times \text{sign}(\prod_v C_v^{\deg(v)})$, and verify this for $n=6$ and $n=7$. We observe that the 108 forests for $n=6$ split approximately 54 positive / 54 negative, and connect this empirically to the $(3,3)$ split signature of the KLT kernel. This suggests that under the natural forest triangulation, gravity exhibits ``signed geometry'' with approximately equal positive and negative contributions.
\end{abstract}

\section{Introduction}

The Amplituhedron program~\cite{ArkaniHamed:2013jha} has revealed that tree-level Yang-Mills amplitudes are canonical forms of positive geometries. A natural question arises: \textit{Does gravity have a positive geometry?}

We provide evidence that the answer is subtle. Under the natural forest triangulation inherited from the Matrix-Tree Theorem, gravity exhibits \textbf{signed geometry}---a structure where the canonical form involves both positive and negative contributions with a characteristic sign pattern. Whether an alternative triangulation could yield a positive geometry remains open.

\subsection{Known Results}

The $n$-point MHV gravity amplitude is given by the Hodges determinant~\cite{Hodges:2011wm}:
\begin{equation}
\mathcal{M}_n^{\text{MHV}} = (-1)^{n-1} \langle ab \rangle^8 \frac{\det(\tilde{L}^{(R)})}{\mathcal{N}_R \prod_{k \notin R} C_k^2}
\end{equation}
where $\tilde{L}$ is a weighted Laplacian. By the Matrix-Tree Theorem~\cite{Chaiken:1982}, this determinant equals a sum over spanning forests~\cite{NSVW:2009}:
\begin{equation}
\det(\tilde{L}^{(R)}) = \sum_{F \in \mathcal{F}_R(K_n)} \prod_{(i,j) \in E(F)} a_{ij}
\end{equation}
where $a_{ij} = w_{ij} C_i C_j$ are the \textbf{MTT edge weights} (defined so that $\tilde{L}_{ij} = -a_{ij}$ for $i \neq j$).

\subsection{Our Contribution}

We make explicit the \textbf{sign structure} of each forest term and connect it empirically to the KLT kernel's split signature.

\section{Conventions}

\textbf{Real kinematic slice.} Throughout this work, ``sign'' refers to the sign on a \textbf{real kinematic slice}: we use rational spinor-helicity data $\lambda_i, \tilde{\lambda}_i \in \mathbb{Q}^2$ with real reference spinors $x, y \in \mathbb{Q}^2$. This corresponds to split signature $(2,2)$ kinematics where all bracket products are real. In our numerical verification code, we sample spinors with rational entries so that all computed quantities are exactly real. For generic complex kinematics, signs are not defined.

\begin{definition}[Hodges Weighted Laplacian]
For $n$ particles with spinor-helicity variables $(\lambda_i, \tilde{\lambda}_i)$ and reference spinors $(x, y)$, define:
\begin{align}
w_{ij} &= \frac{[ij]}{\langle ij \rangle} \quad \text{(kinematic edge weight)} \\
C_i &= \langle i, x \rangle \langle i, y \rangle \quad \text{(reference factor)}
\end{align}
The weighted Laplacian is:
\begin{equation}
\tilde{L}_{ij} = \begin{cases}
-w_{ij} \cdot C_i \cdot C_j & i \neq j \\
\sum_{k \neq i} w_{ik} \cdot C_i \cdot C_k & i = j
\end{cases}
\end{equation}
The MTT edge weight is $a_{ij} := w_{ij} C_i C_j$, so that $\tilde{L}_{ij} = -a_{ij}$ for $i \neq j$.
\end{definition}

\section{Main Result: The Sign Rule}

\begin{theorem}[Sign Rule for Forest Terms]
\label{thm:sign-rule}
For a forest $F \in \mathcal{F}_R(K_n)$ with edge set $E(F)$, the contribution to $\det(\tilde{L}^{(R)})$ is:
\begin{equation}
\text{term}(F) = \prod_{e \in E(F)} a_e = \prod_{e \in E(F)} w_e \cdot \prod_{v \in V} C_v^{\deg_F(v)}
\end{equation}
The sign of this term is:
\begin{equation}
\boxed{\varepsilon(F) = \text{sign}\left(\prod_{e \in E(F)} w_e\right) \times \text{sign}\left(\prod_{v \in V} C_v^{\deg_F(v)}\right)}
\end{equation}
\end{theorem}

\begin{proof}
The All-Minors Matrix-Tree Theorem~\cite{Chaiken:1982} states: for a Laplacian $L$ with $L_{ij} = -a_{ij}$ (off-diagonal) and $L_{ii} = \sum_{j \neq i} a_{ij}$:
\begin{equation}
\det(L^{(R)}) = \sum_{F \in \mathcal{F}_R} \prod_{(i,j) \in E(F)} a_{ij}
\end{equation}
This formula already absorbs the off-diagonal minus signs---each term is a product of the \emph{positive} edge weights $a_{ij}$, not the Laplacian entries.

For our Laplacian, $a_{ij} = w_{ij} C_i C_j$. The $C$ factors combine as:
\begin{equation}
\prod_{(i,j) \in E(F)} C_i C_j = \prod_{v \in V} C_v^{\deg_F(v)}
\end{equation}
since each vertex $v$ appears in exactly $\deg_F(v)$ edges.

Therefore:
\begin{equation}
\text{term}(F) = \prod_{e \in E(F)} w_e \cdot \prod_{v \in V} C_v^{\deg_F(v)}
\end{equation}
and the sign is the product of signs of these factors.
\end{proof}

\begin{remark}[Sign Convention]
Some expositions work with ``signed-edge weights'' $b_{ij} := \tilde{L}_{ij} = -a_{ij}$. If one expands using $b_{ij}$, an extra factor $(-1)^{|E(F)|}$ appears. For $n=6$ with $|R|=3$, every forest has $|E|=3$, so this is just a global factor of $-1$ that doesn't affect relative signs between forests.
\end{remark}

\subsection{Origin of Each Factor}

\begin{center}
\begin{tabular}{|c|c|c|}
\hline
\textbf{Factor} & \textbf{Origin} & \textbf{Dependence} \\
\hline
$\text{sign}(\prod w)$ & Kinematic ratios $[ij]/\langle ij\rangle$ & Kinematic \\
$\text{sign}(\prod C^{\deg})$ & Reference spinors & Gauge (cancels in total) \\
\hline
\end{tabular}
\end{center}

\section{The 50/50 Sign Split}

\begin{observation}[Empirical Sign Split]
\label{obs:empirical-split}
For $n=6$ with $|R|=3$, numerical sampling over 50 generic kinematic configurations (using exact rational arithmetic) shows the 108 spanning forests split with mode $(54, 54)$ positive/negative, occurring in 44\% of samples.
\end{observation}

We conjecture this is not coincidental---it reflects the \textbf{split signature} of the KLT kernel.

\subsection{KLT Kernel Signature}

The KLT relations express gravity as:
\begin{equation}
M_{\text{gravity}} = A_{\text{YM}}^T \cdot S_{\text{KLT}} \cdot \tilde{A}_{\text{YM}}
\end{equation}

\begin{observation}[KLT Signature]
\label{obs:klt-signature}
Numerical computation over 20 generic kinematic configurations (using exact rational arithmetic) shows the KLT kernel matrix for $n=6$ has signature $(3,3)$---three positive and three negative eigenvalues---in all non-degenerate samples.
\end{observation}

\begin{conjecture}[KLT Signature Stability]
On the real kinematic slice (split signature $(2,2)$), away from degeneracies, the $n=6$ KLT kernel has constant signature $(3,3)$.
\end{conjecture}

\begin{theorem}[Mixed Signs are Generic]
\label{thm:mixed-signs}
For 6-point MHV gravity with 3-rooted spanning forests on $K_6$:
\begin{enumerate}
\item[(i)] Of the 15 edges in $K_6$, exactly 3 (the root-root edges) appear in \textbf{zero} forests. The remaining 12 ``forest-relevant'' edges determine all forest signs.
\item[(ii)] Uniform forest signs (all positive or all negative) require all 12 forest-relevant MTT edge weights $a_{ij} = w_{ij} C_i C_j$ to have the same sign. This is satisfied by only $16$ out of $2^{15} = 32768$ edge-sign patterns ($0.049\%$).
\item[(iii)] For generic kinematics where the 12 forest-relevant edge weights $a_{ij}$ are not all the same sign, \textbf{mixed forest signs are forced}.
\end{enumerate}
\end{theorem}

\begin{proof}
\textbf{(i)} A 3-rooted forest with roots $R = \{1,2,3\}$ has each root forming its own tree. Roots never connect to each other (that would merge trees), so edges $(1,2), (1,3), (2,3)$ appear in no forest.

\textbf{(ii)} By exhaustive enumeration: among all $2^{15}$ edge-sign patterns, exactly 8 give all forests positive (all 12 forest edges $+$, any signs on 3 unused edges) and 8 give all forests negative.

\textbf{(iii)} If the 12 forest-relevant MTT edge weights $a_{ij} = w_{ij} C_i C_j$ are not all the same sign, then the 108 forest products cannot all share a sign. Under uniformly random independent edge signs, uniform-sign outcomes occur with probability $16/2^{15} = 0.049\%$.
\end{proof}

\begin{remark}[Edge Weight Signs]
The forest term signs depend on the \textbf{full MTT edge weight} $a_{ij} = w_{ij} C_i C_j$, not on $w_{ij} = [ij]/\langle ij \rangle$ alone. On a fixed real kinematic slice, the reference factors $C_i$ contribute to the overall sign pattern. However, since the total amplitude is reference-independent, the $C$-factor contributions cancel in the full sum.
\end{remark}

\begin{theorem}[Modal Split Under Random Signs]
\label{thm:modal-split}
Consider the 12 forest-relevant edges with i.i.d.\ uniform $\pm 1$ signs. Under this combinatorial model:
\begin{enumerate}
\item[(i)] The expected number of positive forests is exactly $E[\text{positive}] = 54$.
\item[(ii)] The modal split is $(54, 54)$, occurring in $1184/4096 = 28.90625\%$ of the $2^{12}$ sign patterns.
\end{enumerate}
\end{theorem}

\begin{proof}
Each forest has exactly 3 edges. Under i.i.d.\ uniform signs:
\begin{equation}
P(\text{positive}) = P(\text{0 or 2 negative edges}) = \binom{3}{0}\frac{1}{8} + \binom{3}{2}\frac{1}{8} = \frac{4}{8} = \frac{1}{2}
\end{equation}
Therefore $E[\text{positive}] = 108 \times \frac{1}{2} = 54$. By exhaustive enumeration over all $2^{12} = 4096$ sign patterns, $(54,54)$ is the mode with frequency $1184/4096$.
\end{proof}

\begin{remark}[Kinematic Sampling vs.\ Combinatorial Model]
The theorem above concerns the \textbf{combinatorial i.i.d.\ sign model}. Separately, in 50 samples from \textbf{physical kinematics} (random real spinor-helicity data), we observe the $(54,54)$ split in approximately 44\% of samples. The slight difference reflects that physical kinematics do not produce uniformly random edge signs---there are correlations from momentum conservation. Both observations support the conclusion that balanced splits are typical.
\end{remark}

\subsection{The 18-Fold Structure}

The ratio $108/6 = 18$ is exact and structurally significant:
\begin{itemize}
\item \textbf{6 KLT orderings} (permutations of particles $\{3,4,5\}$)
\item \textbf{108 forests} (3-rooted spanning forests on $K_6$)
\item \textbf{Ratio: 18 forests per KLT mode}
\end{itemize}

Similarly, $54/3 = 18$: each positive (or negative) KLT eigenvalue corresponds to approximately 18 positive (or negative) forest terms.

\begin{theorem}[Forest Decomposition by Internal Edges]
\label{thm:forest-decomposition}
The 108 forests decompose by internal edge count as follows:
\begin{center}
\begin{tabular}{|c|c|}
\hline
Internal edges & Count \\
\hline
0 (direct to roots) & 27 \\
1 & 54 \\
2 & 27 \\
\hline
\end{tabular}
\end{center}
\end{theorem}

\begin{proof}
By exhaustive enumeration of all 3-rooted spanning forests on $K_6$. An internal edge connects two non-root vertices; each forest has $n - |R| = 3$ edges total.
\end{proof}

The algebraic map between forests and KLT eigenmodes is \textbf{kinematic-dependent}, not purely combinatorial. Both structures involve the same Mandelstam variables $s_{ij}$, but the exact forest$\to$eigenmode assignment varies with the kinematic point.

\begin{theorem}[Forest Structure by Root Assignment]
\label{thm:forest-structure}
For $n=6$ with roots $R = \{0, 1, 2\}$, each non-root vertex $v \in \{3, 4, 5\}$ is assigned to exactly one root in any spanning forest. The 27 possible root-assignment patterns $(r(3), r(4), r(5))$ decompose as follows:
\begin{enumerate}
\item \textbf{Monochrome patterns} (all three non-roots assigned to same root): 3 patterns, each with 16 forests. Total: 48 forests.
\item \textbf{Mixed patterns} (exactly two non-roots share a root): 18 patterns, each with 3 forests. Total: 54 forests.
\item \textbf{Bijective patterns} (each non-root to a distinct root): 6 patterns, each with 1 forest. Total: 6 forests.
\end{enumerate}
The total is $48 + 54 + 6 = 108$ forests.
\end{theorem}

\begin{proof}
By exhaustive enumeration of all 3-rooted spanning forests on $K_6$. The root assignment for a non-root vertex $v$ is uniquely determined by the connected component containing $v$, which must contain exactly one root. We enumerate all edge subsets of size 3 that form valid forests and classify by root assignment pattern.
\end{proof}

\begin{remark}[The 18-Fold Ratio]
The ratio $108/6 = 18$ is exact and structurally suggestive: there are 6 KLT orderings (permutations of $\{3,4,5\}$) and 108 forests, giving 18 forests ``per ordering.'' However, an explicit canonical map from forests to KLT modes is \textbf{kinematic-dependent} and not purely combinatorial. The question of whether such a map has a natural geometric interpretation remains open.
\end{remark}

\section{Evidence Against Positive Geometry for Gravity}

\begin{definition}[Positive Geometry]
A positive geometry $(X, \Omega)$ satisfies:
\begin{enumerate}
\item $X$ is a bounded region with boundaries
\item $\Omega$ has logarithmic singularities only on boundaries
\item Residues on boundaries give lower-point positive geometries
\end{enumerate}
\end{definition}

Yang-Mills amplitudes satisfy these axioms (Amplituhedron). For gravity, we find:

\begin{observation}[Mixed-Sign Structure]
\label{obs:mixed-sign-structure}
Under the natural forest triangulation of MHV gravity amplitudes, the 108 terms exhibit an intrinsic mixed-sign structure. Observations~\ref{obs:empirical-split} and \ref{obs:klt-signature}, combined with Theorem~\ref{thm:modal-split}, provide computational evidence that approximately 54 positive and 54 negative contributions is typical, correlating with the $(3,3)$ split signature of the KLT kernel.
\end{observation}

\begin{remark}
This does not constitute a proof that \emph{no} positive geometry exists for gravity---such a proof would require demonstrating an obstruction to \emph{all possible} triangulations. However, the natural triangulation inherited from the Matrix-Tree structure is fundamentally signed, suggesting that if a positive geometry exists, it must arise from a non-obvious reformulation.
\end{remark}

\subsection{Signed Geometry}

We propose a generalization:

\begin{definition}[Signed Geometry]
A signed geometry $(X, \Omega_\pm)$ is characterized by:
\begin{enumerate}
\item Full kinematic space $X$ (not a bounded positive region)
\item Signed canonical form: $\Omega_\pm = \sum_{\text{cells}} \varepsilon(\text{cell}) \cdot \omega(\text{cell})$
\item Residues preserve sign structure
\end{enumerate}
\end{definition}

Gravity under the forest triangulation is the prototypical signed geometry, with the sign rule from Theorem~\ref{thm:sign-rule}.

\section{Computational Evidence}

We verify the sign rule and sign split patterns using exact rational arithmetic throughout. This section documents our computational methodology.

\subsection{Sampling Protocol}

\textbf{Kinematic samples:} We generate random spinor-helicity data $(\lambda_i, \tilde{\lambda}_i) \in \mathbb{Q}^2$ satisfying momentum conservation $\sum_i \lambda_i \tilde{\lambda}_i^T = 0$. Spinor components are sampled uniformly from integers in $[-100, 100]$, then the last two $\tilde{\lambda}$ are solved exactly to enforce conservation.

\textbf{Seed handling:} Each sample uses a deterministic seed (sample index $\times$ constant), ensuring reproducibility.

\textbf{Degeneracy handling:} Samples with $\langle ij \rangle = 0$ for any $i \neq j$, or with singular reference factors $C_i = 0$, are detected and skipped. For $n=7$, forests with zero-weight edges (due to $w_{ij} = 0$) are tracked separately.

\subsection{Verification Results}

\begin{center}
\begin{tabular}{|c|c|c|c|c|}
\hline
$n$ & Forests & Modal Split & Sign Rule Accuracy & Samples \\
\hline
6 & 108 & (54, 54) at 44\% & 100\% (2160/2160) & 20 \\
7 & 1029 & (515, 514) at 30\% & 100\% (20580/20580) & 20 \\
\hline
\end{tabular}
\end{center}

\textbf{Note on $n=7$:} The modal split $(515, 514)$ is the closest possible to 50/50 since 1029 is odd. One sample exhibited 597 zero-weight forests due to degenerate kinematics; these are excluded from the split percentage but included in sign-rule verification (where zero-weight terms are skipped consistently).

All verification code is available at: \url{https://github.com/zacharycraig1/Canonical-Positive-Geometry-Twistor-Gravity-Exploration}

\section{Discussion}

\subsection{Relation to Double Copy}

At the formula level, gravity = YM $\times$ YM $\times$ kernel. At the geometry level:
\begin{itemize}
\item Yang-Mills: positive geometry (Amplituhedron)
\item KLT kernel: introduces split signature
\item Gravity: signed geometry (under forest triangulation)
\end{itemize}

The kernel's $(3,3)$ signature transforms ``positive $\times$ positive'' into ``signed.''

\subsection{Signed Factorization}

The sign rule satisfies a crucial property: it is \textbf{multiplicative} under forest decomposition.

\begin{theorem}[Signed Factorization]
\label{thm:factorization}
Let $F$ be a forest on vertex set $V$ that decomposes as $F = F_L \cup F_R$ where $F_L$ and $F_R$ are subforests on disjoint vertex sets $V_L$ and $V_R$ (with $V = V_L \sqcup V_R$). Then:
\begin{equation}
\varepsilon(F) = \varepsilon(F_L) \times \varepsilon(F_R)
\end{equation}
\end{theorem}

\begin{proof}
Since $E(F) = E(F_L) \sqcup E(F_R)$ and the vertex sets are disjoint:
\begin{align}
\prod_{e \in E(F)} w_e &= \prod_{e \in E(F_L)} w_e \times \prod_{e \in E(F_R)} w_e \\
\prod_{v \in V} C_v^{\deg_F(v)} &= \prod_{v \in V_L} C_v^{\deg_{F_L}(v)} \times \prod_{v \in V_R} C_v^{\deg_{F_R}(v)}
\end{align}
The second equality holds because vertices in $V_L$ have $\deg_F(v) = \deg_{F_L}(v)$ (no edges cross to $V_R$), and similarly for $V_R$. Taking signs and applying Theorem~\ref{thm:sign-rule} gives the result.
\end{proof}

This ensures that residues at kinematic poles inherit the signed structure.

\subsection{Uniqueness of Forest Coefficients}

\begin{theorem}[Uniqueness of Monomial Coefficients]
\label{thm:uniqueness}
In the polynomial ring $\mathbb{Z}[a_{ij}]$ with $a_{ij} = w_{ij} C_i C_j$ treated as independent formal variables, the forest expansion
\begin{equation}
\det(\tilde{L}^{(R)}) = \sum_{F \in \mathcal{F}_R} \prod_{e \in E(F)} a_e
\end{equation}
has \textbf{unique} coefficients: each forest $F$ corresponds to a distinct monomial $\prod_{e \in E(F)} a_e$, and all coefficients are $+1$.
\end{theorem}

\begin{proof}
Different forests have different edge sets, so their monomials $\prod_{e \in E(F)} a_e$ are \textbf{distinct as monomials in the independent edge variables} $\{a_{ij}\}$. Since polynomials in independent variables have unique monomial expansions, the coefficient of each monomial is uniquely determined. By the All-Minors Matrix-Tree Theorem, each coefficient is $+1$.

(Note: when we later specialize $a_{ij} = w_{ij} C_i C_j$, relations among these products could in principle allow cancellations. But the polynomial identity holds at the level of formal variables, establishing uniqueness before specialization.)
\end{proof}

\begin{remark}[Sign on Real Slices]
When we specialize to a \textbf{real kinematic slice} (e.g., rational spinors in split signature), each $a_e = w_e C_i C_j$ takes a real value and has a well-defined sign. The ``sign of forest $F$'' is then $\varepsilon(F) = \text{sign}(\prod_{e \in E(F)} a_e)$, which factors as stated in Theorem~\ref{thm:sign-rule}. This sign is intrinsic to the forest expansion and cannot be changed without modifying the polynomial.
\end{remark}

\begin{remark}
This does not prove that no positive geometry exists for gravity---only that the \textbf{forest triangulation in the MTT basis} is intrinsically signed. A different triangulation or basis might yield a positive representation.
\end{remark}

\subsection{Open Questions}

\begin{enumerate}
\item Does the pattern extend to NMHV and loop level?
\item What is the string-theoretic origin of the sign rule?
\item Can twisted cohomology provide a unified framework?
\item Does an alternative triangulation yield positive geometry?
\end{enumerate}

\section{Conclusion}

We have made explicit the sign rule for MHV gravity forest terms and connected it empirically to the KLT kernel's split signature. Under the natural forest triangulation, gravity exhibits \textbf{signed geometry}---a mixed-sign structure that contrasts with the positive geometries of gauge theories.

Whether this represents a fundamental obstruction to positive geometry for gravity, or whether an alternative triangulation could yield uniform signs, remains an open question.

\appendix
\section{Verification Scripts}

The following scripts verify our results:
\begin{itemize}
\item \texttt{tests/test\_oracle\_match.sage}: MTT $\leftrightarrow$ det$(L)$ consistency check (non-circular: compares explicit forest enumeration against Sage matrix determinant)
\item \texttt{src/signed\_geometry/verify\_chy\_sign\_derivation.sage}: Sign rule verification
\item \texttt{src/signed\_geometry/generalize\_n7.sage}: Extension to $n=7$ with full logging
\item \texttt{tests/signed\_geometry\_verification.sage}: Full test suite (15/15 pass)
\end{itemize}

\bibliographystyle{unsrt}
\begin{thebibliography}{10}

\bibitem{ArkaniHamed:2013jha}
N.~Arkani-Hamed and J.~Trnka,
``The Amplituhedron,''
JHEP \textbf{10} (2014) 030,
arXiv:1312.2007.

\bibitem{Hodges:2011wm}
A.~Hodges,
``A simple formula for gravitational MHV amplitudes,''
JHEP \textbf{01} (2013) 059,
arXiv:1108.2227.

\bibitem{Chaiken:1982}
S.~Chaiken,
``A combinatorial proof of the all minors matrix tree theorem,''
SIAM J. Algebraic Discrete Methods \textbf{3} (1982) 319.

\bibitem{NSVW:2009}
D.~Nguyen, M.~Spradlin, A.~Volovich, and C.~Wen,
``The tree formula for MHV graviton amplitudes,''
JHEP \textbf{07} (2009) 045,
arXiv:0907.2276.

\bibitem{KLT:1985}
H.~Kawai, D.~Lewellen, and S.-H.~Tye,
``A relation between tree amplitudes of closed and open strings,''
Nucl. Phys. B \textbf{269} (1986) 1.

\bibitem{Mizera:2017}
S.~Mizera,
``Scattering Amplitudes from Intersection Theory,''
Phys. Rev. Lett. \textbf{120} (2018) 141602,
arXiv:1711.00469.

\end{thebibliography}

\end{document}
