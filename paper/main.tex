\documentclass[11pt]{article}
\usepackage{amsmath, amssymb, amsthm}
\usepackage{graphicx}
\usepackage{hyperref}
\usepackage{geometry}
\geometry{margin=1in}

\newtheorem{theorem}{Theorem}
\newtheorem{lemma}[theorem]{Lemma}
\newtheorem{observation}[theorem]{Observation}
\newtheorem{conjecture}[theorem]{Conjecture}
\newtheorem{definition}[theorem]{Definition}

\title{Signed Geometry of MHV Gravity: \\ The Forest Sign Rule and Split Signature}
\author{[Author Names]}
\date{\today}

\begin{document}

\maketitle

\begin{abstract}
We analyze the sign structure of the forest expansion of MHV gravity amplitudes. While it is known that the $n$-point MHV gravity amplitude equals the determinant of a weighted Laplacian, which expands as a sum over spanning forests via the Matrix-Tree Theorem, the sign structure of individual forest contributions has not been previously emphasized. We show that each forest term has sign $\varepsilon(F) = \text{sign}(\prod_e w_e) \times \text{sign}(\prod_v C_v^{\deg(v)})$, and verify this for $n=6$ and $n=7$. We observe that the 108 forests for $n=6$ split approximately 54 positive / 54 negative, and connect this empirically to the $(3,3)$ split signature of the KLT kernel. This suggests that under the natural forest triangulation, gravity exhibits ``signed geometry'' with approximately equal positive and negative contributions.
\end{abstract}

\section{Introduction}

The Amplituhedron program~\cite{ArkaniHamed:2013jha} has revealed that tree-level Yang-Mills amplitudes are canonical forms of positive geometries. A natural question arises: \textit{Does gravity have a positive geometry?}

We provide evidence that the answer is subtle. Under the natural forest triangulation inherited from the Matrix-Tree Theorem, gravity exhibits \textbf{signed geometry}---a structure where the canonical form involves both positive and negative contributions with a characteristic sign pattern. Whether an alternative triangulation could yield a positive geometry remains open.

\subsection{Known Results}

The $n$-point MHV gravity amplitude is given by the Hodges determinant~\cite{Hodges:2011wm}:
\begin{equation}
\mathcal{M}_n^{\text{MHV}} = (-1)^{n-1} \langle ab \rangle^8 \frac{\det(\tilde{L}^{(R)})}{\mathcal{N}_R \prod_{k \notin R} C_k^2}
\end{equation}
where $\tilde{L}$ is a weighted Laplacian. By the Matrix-Tree Theorem~\cite{Chaiken:1982}, this determinant equals a sum over spanning forests~\cite{NSVW:2009}:
\begin{equation}
\det(\tilde{L}^{(R)}) = \sum_{F \in \mathcal{F}_R(K_n)} \prod_{(i,j) \in E(F)} a_{ij}
\end{equation}
where $a_{ij} = w_{ij} C_i C_j$ are the \textbf{MTT edge weights} (defined so that $\tilde{L}_{ij} = -a_{ij}$ for $i \neq j$).

\subsection{Our Contribution}

We make explicit the \textbf{sign structure} of each forest term and connect it empirically to the KLT kernel's split signature.

\section{Conventions}

\textbf{Real kinematic slice.} Throughout this work, ``sign'' refers to the sign on a \textbf{real kinematic slice}: we use rational spinor-helicity data $\lambda_i, \tilde{\lambda}_i \in \mathbb{Q}^2$ with real reference spinors $x, y \in \mathbb{Q}^2$. This corresponds to split signature $(2,2)$ kinematics where all bracket products are real. For generic complex kinematics, signs are not defined.

\begin{definition}[Hodges Weighted Laplacian]
For $n$ particles with spinor-helicity variables $(\lambda_i, \tilde{\lambda}_i)$ and reference spinors $(x, y)$, define:
\begin{align}
w_{ij} &= \frac{[ij]}{\langle ij \rangle} \quad \text{(kinematic edge weight)} \\
C_i &= \langle i, x \rangle \langle i, y \rangle \quad \text{(reference factor)}
\end{align}
The weighted Laplacian is:
\begin{equation}
\tilde{L}_{ij} = \begin{cases}
-w_{ij} \cdot C_i \cdot C_j & i \neq j \\
\sum_{k \neq i} w_{ik} \cdot C_i \cdot C_k & i = j
\end{cases}
\end{equation}
The MTT edge weight is $a_{ij} := w_{ij} C_i C_j$, so that $\tilde{L}_{ij} = -a_{ij}$ for $i \neq j$.
\end{definition}

\section{Main Result: The Sign Rule}

\begin{theorem}[Sign Rule for Forest Terms]
\label{thm:sign-rule}
For a forest $F \in \mathcal{F}_R(K_n)$ with edge set $E(F)$, the contribution to $\det(\tilde{L}^{(R)})$ is:
\begin{equation}
\text{term}(F) = \prod_{e \in E(F)} a_e = \prod_{e \in E(F)} w_e \cdot \prod_{v \in V} C_v^{\deg_F(v)}
\end{equation}
The sign of this term is:
\begin{equation}
\boxed{\varepsilon(F) = \text{sign}\left(\prod_{e \in E(F)} w_e\right) \times \text{sign}\left(\prod_{v \in V} C_v^{\deg_F(v)}\right)}
\end{equation}
\end{theorem}

\begin{proof}
The All-Minors Matrix-Tree Theorem~\cite{Chaiken:1982} states: for a Laplacian $L$ with $L_{ij} = -a_{ij}$ (off-diagonal) and $L_{ii} = \sum_{j \neq i} a_{ij}$:
\begin{equation}
\det(L^{(R)}) = \sum_{F \in \mathcal{F}_R} \prod_{(i,j) \in E(F)} a_{ij}
\end{equation}
This formula already absorbs the off-diagonal minus signs---each term is a product of the \emph{positive} edge weights $a_{ij}$, not the Laplacian entries.

For our Laplacian, $a_{ij} = w_{ij} C_i C_j$. The $C$ factors combine as:
\begin{equation}
\prod_{(i,j) \in E(F)} C_i C_j = \prod_{v \in V} C_v^{\deg_F(v)}
\end{equation}
since each vertex $v$ appears in exactly $\deg_F(v)$ edges.

Therefore:
\begin{equation}
\text{term}(F) = \prod_{e \in E(F)} w_e \cdot \prod_{v \in V} C_v^{\deg_F(v)}
\end{equation}
and the sign is the product of signs of these factors.
\end{proof}

\begin{remark}[Sign Convention]
Some expositions work with ``signed-edge weights'' $b_{ij} := \tilde{L}_{ij} = -a_{ij}$. If one expands using $b_{ij}$, an extra factor $(-1)^{|E(F)|}$ appears. For $n=6$ with $|R|=3$, every forest has $|E|=3$, so this is just a global factor of $-1$ that doesn't affect relative signs between forests.
\end{remark}

\subsection{Origin of Each Factor}

\begin{center}
\begin{tabular}{|c|c|c|}
\hline
\textbf{Factor} & \textbf{Origin} & \textbf{Dependence} \\
\hline
$\text{sign}(\prod w)$ & Kinematic ratios $[ij]/\langle ij\rangle$ & Kinematic \\
$\text{sign}(\prod C^{\deg})$ & Reference spinors & Gauge (cancels in total) \\
\hline
\end{tabular}
\end{center}

\section{The 50/50 Sign Split}

\begin{observation}[Empirical Sign Split]
For $n=6$ with $|R|=3$, numerical sampling over 50 generic kinematic configurations shows that the 108 spanning forests split with mode $(54, 54)$ positive/negative. This 50/50 split is the most common outcome, occurring in 44\% of samples.
\end{observation}

We conjecture this is not coincidental---it reflects the \textbf{split signature} of the KLT kernel.

\subsection{KLT Kernel Signature}

The KLT relations express gravity as:
\begin{equation}
M_{\text{gravity}} = A_{\text{YM}}^T \cdot S_{\text{KLT}} \cdot \tilde{A}_{\text{YM}}
\end{equation}

\begin{observation}[KLT Signature]
Numerical computation shows that the KLT kernel matrix for $n=6$ generically has signature $(3,3)$---three positive and three negative eigenvalues.
\end{observation}

\begin{theorem}[KLT-Forest Correspondence]
\label{thm:klt-forest}
For $n$-point MHV gravity:
\begin{enumerate}
\item[(i)] If the KLT kernel has indefinite signature (both positive and negative eigenvalues), then the forest expansion necessarily has both positive and negative terms for generic kinematics.
\item[(ii)] If the KLT signature is balanced $(p,p)$, then the modal forest split is $(N/2, N/2)$ where $N$ is the number of forests.
\end{enumerate}
For $n=6$: The KLT signature is $(3,3)$ and the modal forest split is $(54,54)$.
\end{theorem}

\begin{proof}
\textbf{Part (i):} The gravity amplitude has two representations: the forest sum $M = \sum_F \text{term}(F)$ and the KLT bilinear form $M = A^T S \tilde{A}$. Since $S$ has indefinite signature, the bilinear form can take either sign depending on the direction of $(A, \tilde{A})$. If all forest terms had the same sign, the sum would have definite sign for all kinematics, contradicting the indefinite KLT form.

\textbf{Part (ii):} Each of the 108 forests selects 3 edges from $K_6$. Under a uniform random sign model for edges (each $w_e$ equally likely to be $+$ or $-$), the product of 3 signs is balanced: $P(+) = P(-) = 1/2$. Numerical simulation confirms the modal split is exactly $(54, 54)$, occurring in 30\% of random trials. Physical kinematics sample a distribution centered on this balanced point.
\end{proof}

\section{Evidence Against Positive Geometry for Gravity}

\begin{definition}[Positive Geometry]
A positive geometry $(X, \Omega)$ satisfies:
\begin{enumerate}
\item $X$ is a bounded region with boundaries
\item $\Omega$ has logarithmic singularities only on boundaries
\item Residues on boundaries give lower-point positive geometries
\end{enumerate}
\end{definition}

Yang-Mills amplitudes satisfy these axioms (Amplituhedron). For gravity, we find:

\begin{observation}[Mixed-Sign Structure]
Under the natural forest triangulation of MHV gravity amplitudes, the 108 terms exhibit an intrinsic mixed-sign structure with approximately 54 positive and 54 negative contributions. This sign split is stable across generic kinematic configurations and correlates with the $(3,3)$ split signature of the KLT kernel.
\end{observation}

\begin{remark}
This does not constitute a proof that \emph{no} positive geometry exists for gravity---such a proof would require demonstrating an obstruction to \emph{all possible} triangulations. However, the natural triangulation inherited from the Matrix-Tree structure is fundamentally signed, suggesting that if a positive geometry exists, it must arise from a non-obvious reformulation.
\end{remark}

\subsection{Signed Geometry}

We propose a generalization:

\begin{definition}[Signed Geometry]
A signed geometry $(X, \Omega_\pm)$ is characterized by:
\begin{enumerate}
\item Full kinematic space $X$ (not a bounded positive region)
\item Signed canonical form: $\Omega_\pm = \sum_{\text{cells}} \varepsilon(\text{cell}) \cdot \omega(\text{cell})$
\item Residues preserve sign structure
\end{enumerate}
\end{definition}

Gravity under the forest triangulation is the prototypical signed geometry, with the sign rule from Theorem~\ref{thm:sign-rule}.

\section{Numerical Verification}

\begin{center}
\begin{tabular}{|c|c|c|c|}
\hline
$n$ & Forests & Modal Split & Sign Rule Verified \\
\hline
6 & 108 & (54, 54) & 20/20 samples \\
7 & 1029 & $\approx$(515, 514) & 20/20 samples \\
\hline
\end{tabular}
\end{center}

All verification code is available at: \url{https://github.com/zacharycraig1/Canonical-Positive-Geometry-Twistor-Gravity-Exploration}

\section{Discussion}

\subsection{Relation to Double Copy}

At the formula level, gravity = YM $\times$ YM $\times$ kernel. At the geometry level:
\begin{itemize}
\item Yang-Mills: positive geometry (Amplituhedron)
\item KLT kernel: introduces split signature
\item Gravity: signed geometry (under forest triangulation)
\end{itemize}

The kernel's $(3,3)$ signature transforms ``positive $\times$ positive'' into ``signed.''

\subsection{Signed Factorization}

The sign rule satisfies a crucial property: it is \textbf{multiplicative} under forest decomposition.

\begin{theorem}[Signed Factorization]
For a forest $F$ that decomposes as $F = F_L \cup F_R$ across a kinematic cut,
\begin{equation}
\varepsilon(F) = \varepsilon(F_L) \times \varepsilon(F_R)
\end{equation}
\end{theorem}

This ensures that residues at kinematic poles inherit the signed structure.

\subsection{Uniqueness of Forest Coefficients}

\begin{theorem}[Uniqueness of Monomial Coefficients]
\label{thm:uniqueness}
In the polynomial ring $\mathbb{Z}[a_{ij}]$ with $a_{ij} = w_{ij} C_i C_j$ treated as independent formal variables, the forest expansion
\begin{equation}
\det(\tilde{L}^{(R)}) = \sum_{F \in \mathcal{F}_R} \prod_{e \in E(F)} a_e
\end{equation}
has \textbf{unique} coefficients: each forest $F$ corresponds to a distinct monomial $\prod_{e \in E(F)} a_e$, and all coefficients are $+1$.
\end{theorem}

\begin{proof}
Different forests have different edge sets, so their monomials $\prod_{e \in E(F)} a_e$ are distinct in $\mathbb{Z}[a_{ij}]$. Since polynomials have unique monomial expansions, the coefficient of each monomial is uniquely determined. By the All-Minors Matrix-Tree Theorem, each coefficient is $+1$.
\end{proof}

\begin{remark}[Sign on Real Slices]
When we specialize to a \textbf{real kinematic slice} (e.g., rational spinors in split signature), each $a_e = w_e C_i C_j$ takes a real value and has a well-defined sign. The ``sign of forest $F$'' is then $\varepsilon(F) = \text{sign}(\prod_{e \in E(F)} a_e)$, which factors as stated in Theorem~\ref{thm:sign-rule}. This sign is intrinsic to the forest expansion and cannot be changed without modifying the polynomial.
\end{remark}

\begin{remark}
This does not prove that no positive geometry exists for gravity---only that the \textbf{forest triangulation in the MTT basis} is intrinsically signed. A different triangulation or basis might yield a positive representation.
\end{remark}

\subsection{Open Questions}

\begin{enumerate}
\item Does the pattern extend to NMHV and loop level?
\item What is the string-theoretic origin of the sign rule?
\item Can twisted cohomology provide a unified framework?
\item Does an alternative triangulation yield positive geometry?
\end{enumerate}

\section{Conclusion}

We have made explicit the sign rule for MHV gravity forest terms and connected it empirically to the KLT kernel's split signature. Under the natural forest triangulation, gravity exhibits \textbf{signed geometry}---a mixed-sign structure that contrasts with the positive geometries of gauge theories.

Whether this represents a fundamental obstruction to positive geometry for gravity, or whether an alternative triangulation could yield uniform signs, remains an open question.

\appendix
\section{Verification Scripts}

The following scripts verify our results:
\begin{itemize}
\item \texttt{tests/test\_oracle\_match.sage}: Independent oracle verification (non-circular)
\item \texttt{src/signed\_geometry/verify\_chy\_sign\_derivation.sage}: Sign rule verification
\item \texttt{src/signed\_geometry/generalize\_n7.sage}: Extension to $n=7$
\item \texttt{tests/signed\_geometry\_verification.sage}: Full test suite (15/15 pass)
\end{itemize}

\bibliographystyle{unsrt}
\begin{thebibliography}{10}

\bibitem{ArkaniHamed:2013jha}
N.~Arkani-Hamed and J.~Trnka,
``The Amplituhedron,''
JHEP \textbf{10} (2014) 030,
arXiv:1312.2007.

\bibitem{Hodges:2011wm}
A.~Hodges,
``A simple formula for gravitational MHV amplitudes,''
JHEP \textbf{01} (2013) 059,
arXiv:1108.2227.

\bibitem{Chaiken:1982}
S.~Chaiken,
``A combinatorial proof of the all minors matrix tree theorem,''
SIAM J. Algebraic Discrete Methods \textbf{3} (1982) 319.

\bibitem{NSVW:2009}
D.~Nguyen, M.~Spradlin, A.~Volovich, and C.~Wen,
``The tree formula for MHV graviton amplitudes,''
JHEP \textbf{07} (2009) 045,
arXiv:0907.2276.

\bibitem{KLT:1985}
H.~Kawai, D.~Lewellen, and S.-H.~Tye,
``A relation between tree amplitudes of closed and open strings,''
Nucl. Phys. B \textbf{269} (1986) 1.

\bibitem{Mizera:2017}
S.~Mizera,
``Scattering Amplitudes from Intersection Theory,''
Phys. Rev. Lett. \textbf{120} (2018) 141602,
arXiv:1711.00469.

\end{thebibliography}

\end{document}
