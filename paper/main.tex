\documentclass{article}
\usepackage{amsmath, amssymb}
\usepackage{graphicx}
\usepackage{hyperref}

\title{The Positive Geometry of $n$-point MHV Gravity: \\ Spanning Forests and Weighted Laplacians}
\author{Cursor Agent}
\date{\today}

\begin{document}

\maketitle

\begin{abstract}
We present a manifestly positive geometric construction for the $n$-point MHV gravity amplitude. We show that the amplitude is given exactly by the principal minor of a Reference-Weighted Laplacian matrix, which computes a sum over 3-rooted spanning forests of the complete graph $K_n$. This object defines a generalized permutohedron in kinematic space. We verify the formula analytically and numerically for $n=6,7$ and prove its reference independence.
\end{abstract}

\section{Introduction}
Recent developments in scattering amplitudes have revealed deep connections to positive geometries...

\section{The Weighted Laplacian}
We define the weighted Laplacian $\tilde{L}$ with entries...

\section{Main Result}
The $n$-point MHV gravity amplitude is:
\begin{equation}
\mathcal{M}_n = (-1)^{n-1} \langle ab \rangle^8 \frac{\det(\tilde{L}^{(R)})}{\prod_{k \notin R} C_k^2 |\mathcal{N}(R)|^2}
\end{equation}

\section{Combinatorial Interpretation}
By the All-Minors Matrix-Tree Theorem, the determinant counts 3-component rooted forests...

\section{Conclusion}
This work establishes the forest polytope as the geometric heart of gravity...

\appendix
\section{Verification Suite}
All results are reproducible using the provided scripts in the \texttt{repro/} directory.

\bibliographystyle{plain}
\bibliography{refs}

\end{document}







